\documentclass[12pt]{article}
\usepackage[utf8]{inputenc}
\usepackage[english]{babel}
 
\usepackage{sectsty}
\usepackage{multicol}
\usepackage{apacite}
\setlength{\columnsep}{1cm}
\usepackage[margin=0.5in]{geometry} % don't know what size margins to use

\sectionfont{\fontsize{12}{15}\selectfont}
 

\begin{document}
    \begin{center}
        \textbf{Evaluation of Classifiers and Feature Selection on Spam Filtering} 
    \end{center}

    \begin{center}
        Griffin Dunn and William Thompson
    \end{center}

    \textbf{Abstract:} 
        In this paper we explore the effects of extracting features that are not solely based on word frequency.

    \begin{multicols}{2}
        \section{Introduction}
            

        \section{Dataset Selection}
            The correct selection of a dataset was crucial to the testing of 
            different types of features. There were many characteristics that were
            on our "wishlist." Such as having the freedom of processing 
            the raw text ourselves and determining what features may be important, 
            and not have the data preproccesed or turned into vectors.
            Next we wanted data that was conversational, that didn't use too much 
            slang or leetspeak and was in general proper English. The size of the
            dataset needed to be fairly large, at least 1000, for frequency analysis.
             Lastly, we would like a dataset that was referenced in multiple papers to make comparisons
            of our results to others. So we shortlisted a few datasets that fulfilled
            some of these characteristics.

            The UCI SMS Spam Dataset is a collection of conversational text messages
            origninating from a variety of sources.
        \section{Data Preparation}
            Once data was loaded in we started by running the data the data through
            the Python Standard Library's mail parser, which striped the mail headers
            and put them into a map for separate analysis. It also handled multipart
            data as files were encoded into the file in base64. Without using a mail
            parser it would be difficult to just get the body of the text which
            would be used for feature extraction. Other papers left this data in the
            email, possibly because this information is hard to extract and the tools
            weren't as easily accessable when their implementation was written. We
            decided to strip this data out as to not add noise and also because many 
            headers were left on from other spam checkers, identifying whether or not
            it believed the message was spam or ham.
        \section{Feature Extraction}

        \section{Classifier Selection}

        \section{Classifier Tuning}

        \section{Results}

        \section{Evaluation}

        \section{Conclusion}

        \section{References}

    \end{multicols}
    %\bibliographystyle{apacite}
    %\bibliography{spam.bib}
\end{document}